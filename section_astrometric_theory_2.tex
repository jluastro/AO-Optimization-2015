It is not sufficient in the case of variable PSFs to estimate the empirical centroid on each individual PSF as the astrometric reference because it is unknown what effect the individual shape has on the centroid measurement (additionally, this depends on the particular method for calculating the center of light). As a result, the difference between the true position and the measured center is different for each of the PSFs, and contributes to the systematics in the relative astrometry.

It is essential to maintain the positional relation between the individual modeled PSFs. In our sequence of analysis steps this means to first extract an 'on-axis' estimate (no atmospheric anisoplanatism, no instrumental aberrations other than at the image-sharpening position on the detector) and to center this estimate with some centroiding algorithm. {\bf Centering here means to shift the empirical reference point within the PSF to the pixel that defines the astrometric reference for the PSF-fitting algorithm, in our case the central pixel of the PSF array.} In the next steps of propagating the PSFs to off-axis positions the relation between the true position and the central pixel has to be static and part of our astrometric calibration. In order to keep this relation invariant with time we have to understand the effect that optical aberrations have on the difference between true position and the reference point calculated from the PSF.  

\subsection{The relation between the true position and the center of light}

In the following we mainly summarize key results of Mahajan (1985), 'The Line of Sight of an Aberrated Optical System', and apply them to our strategy for astrometric measurements with field variable PSFs.

  