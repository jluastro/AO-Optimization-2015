In year 4 we made significant progress in developing a software suite that allows us to apply the PSF modeling tools to on-sky Galactic Center data. Builing on the prototype PSF-extraction presented last year we fully implemented PSF-modeling, PSF-extraction, and PSF-fitting routines. Furthermore, we gathered several versions of StarFinder1.6 with improvements in PSF extraction and post-processing from various sources and implemented and tested all modifications.

As a result StarFinder2.0 offers:
\begin{enumerate}
\item Full functionality of SF1.6 and thus full backwards compatibility.
\item PSF-extraction and -fitting with variable PSFs, both atmospheric and instrumental. The instrumental interface uses phase-maps as the input data and can be used for introducing any kind of phase aberrations (e.g., for simulations).
\item Improved handling of secondary sources (stars close to a PSF reference source) in the PSF-extraction.
\item Improved background and noise subtraction/clipping for PSF-estimates.
\item A new GC analysis wrapper for StarFinder that makes it possible to test the psf-modeling against on-sky data of crowded fields.
\item Version control via GIT repository.
\item Unit and regression tests.
\end{enumerate}
  
  
  