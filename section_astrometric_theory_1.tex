\section{Astrometric Theory}
\label{sec:astrometric_theory}

\begin{itemize}
\item Multiple PSFs don't solve everything... you need to be careful about how you centroid PSFs on a grid relative to each other. 
\item The reference point problem - What is the difference between the difference of the true position of a star (or true separations between two stars on the sky) vs. some reference point in the PSF such as the large-box centroid, small-box centroid, or maximum-flux pixel. 
\item Aberrations introduce arbitrary shifts of these PSF/OTF derived (measurable) quantities w.r.t. the true position (or separation).
\end{itemize}
  
We outline a strategy for relative astrometry with spatially variable adaptive optics point spread functions (PSFs). We summarize the necessary definitions and underlying assumptions, and present a detailed methodology for maintaining the positional relation between the individual PSFs when modeling the spatially variable effects due to atmospheric anisoplanatism and instrumental aberrations. This methodology, which properly defines the astrometric center of each PSF, is the centerpiece of an improved astrometry that takes into account the spatial variability of the PSF-shapes.

\subsection{The reference point problem}

The problem discussed in this short outline can be described best with the sketch shown in Fig.~\ref{fig:ast_theory_psf}. 

In the case of a constant PSF the astrometric reference point is usually calculated from the (empirical) center of light (centroid) calculated on the empirical PSF with finite support (in our case estimated from the science data itself). In general, this center coincides with neither the true mathematical center of light (calculated as an integral over an infinite domain), nor the true position of the object (more on the definition of the true position in the following section). On the other hand, since the PSF is (assumed to be) constant, the difference between true position and centroid - although unknown - is always the same for all positions in the field of view, and thus relative astrometric measurements remain unaffected.

{\bf In order to understand the problems occurring in the case of astrometry with variable PSFs it is helpful to think in terms of three different coordinate systems: one with its origin at the true position, one with its origin at a reference point calculated from the intensity distribution of the PSF itself (e.g., the empirical or the mathematical centroid), and one with its origin at the central pixel of the array used to represent the PSF for the PSF fitting algorithm (assumed to be the true position within Starfinder).}
