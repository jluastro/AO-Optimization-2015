\section{Introduction}

High-resolution observations of stars orbiting close to our Galaxy's supermassive black hole will enable unique tests of general relativity in the strong gravity regime. Adaptive optics (AO) technology currently provides the sharpest infrared view of the Galactic Center \cite{Ghez_2005}. However, the potential of this technology has not been fully exploited in terms of astrometric and photometric precision due to our incomplete knowledge of the temporal and spatial variations in the point spread function (PSF). In 2011, the Galactic Center Group at UCLA embarked on a project to optimize the extraction of AO astrometry and photometry by incorporating measurements of the Earth's atmospheric turbulence profile and instrumental aberrations to build a model of the PSF spatial variations in every AO image obtained with the NIRC2 instrument on the Keck II telescope. This project, entitled \textbf{AO Optimization} is funded by the W. M. Keck Foundation. The ultimate objective of the AO Optimization project is to apply these methods to past and future observations of stars near the supermassive black hole (SMBH) in order to use their orbits to test General relativity. The star closest to the SMBH (S0-2) will have a periapse passage, between 2016-2020; thus, it is essential that we have AO Optimization tools in place in time for these observations. 

The following document to the external advisory committee (EAC) is intended as a progress report on activities during the period Summer 2014 - July 2015 (Year 4). We first give a project overview followed by sections on the instrumental aberration maps, atmospheric modeling, astrometric theory, tool development for extracting astrometry and photometry with a spatially variable PSF, and testing AO Optimization on simulated and on-sky data. We present the status of our dissemination efforts and an extension of the AO Optimization project to OSIRIS funded by the Moore foundation. Finally, we discuss our plans for the final year of the AO Optimization project (Year 5).   

\subsection{Galactic Center Science}


\subsection{EAC Recommendations}

\subsection{Year 4 Overview}
The following report is written for the External Advisory Committee (EAC) visit in July, 2015. This report contains a status update on the AO Optimization project including completed milestones in Year 4 (2014-07-01 to 2015-07-01) and upcoming milestones for the final Year 5. As a reminder, the AO Optimization project consists of the following high-level goals:
\begin{enumerate}
\item Map NIRC2 instrumental aberrations.
\item Develope algorithms to predict atmospheric-induced aberrations from MASS/DIMM data.
\item Update NIRC2 distortion solution.
\item Upgrade Starfinder 2.0 to extract on-axis PSF from image with spatially varying PSF.
\item Upgrade Starfinder 2.0 to extract stellar astrometry and photometry using varying PSFs.
\item Apply AO Optimization to Galactic Center astrometry.
\item Disseminate AO Optimization to a broader community.
\end{enumerate}




    
    
    
    
    
    
    
  
  
  
  
  