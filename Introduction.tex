\section{Introduction}

High-resolution observations of stars orbiting close to our Galaxy's supermassive black hole will enable unique tests of general relativity in the strong gravity regime. Adaptive optics (AO) technology currently provides the sharpest infrared view of the Galactic Center \cite{Ghez_2005}. However, the potential of this technology has not been fully exploited in terms of astrometric and photometric precision due to our incomplete knowledge of the temporal and spatial variations in the point spread function (PSF). In 2011, the Galactic Center Group at UCLA embarked on a project to optimize the extraction of AO astrometry and photometry by incorporating measurements of the Earth's atmospheric turbulence profile and instrumental aberrations to build a model of the PSF spatial variations in every AO image obtained with the NIRC2 instrument on the Keck II telescope. This project, entitled \textbf{AO Optimization} is funded by the W. M. Keck Foundation. The ultimate objective of the AO Optimization project is to apply these methods to past and future observations of stars near the supermassive black hole (SMBH) in order to use their orbits to test General relativity. The star closest to the SMBH (S0-2) will have a periapse passage, between 2016-2020; thus, it is essential that we have AO Optimization tools in place in time for these observations. 

The following document to the external advisory committee (EAC) is intended as a progress report on activities during the period Summer 2014 - July 2015 (Year 4). We first give a project overview (\S\ref{sec:overview}) followed by sections on the instrumental aberration maps (\S\ref{sec:instrument}), atmospheric modeling, astrometric theory (\S\ref{sec:astrometry_theory}), tool development for extracting astrometry and photometry with a spatially variable PSF (\S\ref{sec:tools}), and testing AO Optimization on simulated and on-sky data (\S\ref{sec:testing}). We present the status of our dissemination efforts in \S\ref{sec:dissemination} and an extension of the AO Optimization project to OSIRIS funded by the Moore foundation in \S\ref{sec:osiris}. Finally, we discuss our plans for the final year of the AO Optimization project (Year 5) in \S\ref{sec:year5}.





    
    
    
    
    
    
    
  
  
  
  
  