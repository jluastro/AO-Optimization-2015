\section{Year 5 Plans}
\label{sec:year5}

We present below a detailed breakdown of the goals for Year 5. 
\begin{enumerate}
\item[Goal 1.] Develop algorithms to predict atmospheric-induced aberrations
  from MASS/DIMM data for both natural and laser guide star
  observations.
  \begin{itemize}
  \item Implement new IDL interface to atmospheric code to improve
    efficiency of STF 2.0.
  \item Fix bugs that currently limit PSF size to 181$\times$181
    pixels.
  \item Finalize installation documentation for atmospheric code and
    dependencies (e.g. Arroyo). 
  \item Finalize documentation and examples.
  \item Release package (v1.0).
  \end{itemize}

\item[Goal 2.] Map NIRC2 instrumental aberrations that effect the PSF
  shape.
  \begin{itemize}
  \item Complete last 30\% of the ``new'' grid of phase diversity
    data (post-hardware upgrade).
  \item Deliver the ``new'' aberration map.
  \end{itemize}

\item[Goal 3.] Update the NIRC2 distortion solution utilizing a spatially
  variable PSF.
  \begin{itemize}
  \item Run STF 2.0 on globular cluster data to create starlists using
    both a single PSF and a spatially varying PSF.
  \item Derive a NIRC2 distortion solutions using these starlists.
  \item Deliver the single-PSF NIRC2 distortion solution to Keck to
    upgrade the existing website.
  \item Package the variable-PSF NIRC2 distortion solution with
    Starfinder 2.0 and AO Optimization.
  \item Modify our NIRC2 reduction pipeline to utilize the new
    distortion solution.
  \end{itemize}

\item[Goal 4.] Upgrade Starfinder 2.0 to extract an on-axis PSF from images with
  a spatially varying PSF due to atmospheric and instrumental aberrations.
  \begin{itemize}
  \item Incorporate final ``old'' and ``new'' NIRC2 aberration maps in code.
  \item Test on simulated images with spatially varying PSFs and
    quantify residuals under different conditions. 
  \item Fix minor bugs.
  \end{itemize}

\item[Goal 5.] Upgrade Starfinder 2.0 to extract stellar astrometry and
  photometry using spatially varying PSFs.
  \begin{itemize}
  \item Test on simulated images with a smoothly varying PSF to
    determine optimal number of grid points for the STF 2.0 grid of
    PSFs.
  \item Test analysis on single images vs. combined images.
  \item Fix minor bugs.
  \item Add documentation for STF 2.0 upgrades, installation,
    wrappers, and example scripts.
  \end{itemize}

\item[Goal 6.] Apply AO Optimization tools to Galactic Center and demonstrate
  astrometric gains.
  \begin{itemize}
  \item Simulations
    \begin{itemize}
    \item Single-epoch image stack error budget.
    \item Maser mosaic and astromtric reference frame error budget.
    \item S0-2 full-orbit error budget.
    \end{itemize}
  \item Observations
    \begin{itemize}
    \item Test single-epoch image stack (small dither) under variable
      conditions. 
    \item Test single-epoch image mosaic (large dither) to validate
      improvements in astrometric accuracy. 
    \item Test image stacks from adjacent nights to examine
      astrometric precison/accuracy in different atmospheric conditions.
    \item Re-analyze all Galactic Center AO data with MASS/DIMM
      (central pointings, maser mosaics) with STF 2.0.
    \item Derive new reference frame.
    \item Measure new proper motions and orbits.
    \item Derive new black hole mass and distance.
    \item Compare results from AO Optimization to previous results.
    \end{itemize}
  \end{itemize}

\item[Goal 7.] Disseminate AO Optimization tools to a broader
  community.
  \begin{itemize}
  \item Publish papers on the complete AO Optimization algorithms,
    centroid handling, and Starfinder 2.0 upgrades. 
  \item Publish papers on Galactic Center science gains with AO Optimization.
  \item Release complete software package, including STF 2.0, NIRC2
    distortion solution for variable PSF, and atmospheric code to
    alpha-testers (i.e. Keck, Moore AO Opt).
  \end{itemize}
\end{enumerate}
