\section{Dissemination and Broader Impacts}
\label{sec:dissemination}

The AO Optimization team has taken extensive steps during Year 4 to 
package the project software into a cohesive tool set that can
efficiently be installed and edited by the entire AO Optimization
team. The AO Optimization tools consist of two distinct packages:
\begin{itemize}
\item Atmospheric Code
\item Starfinder 2.0 (plus wrappers)
\end{itemize}
where the atmospheric code is considered a third party dependency for
Starfinder 2.0 (Figure \ref{fig:tool_schematic}).

The atmospheric code repository has been migrated from tOSC to
bitbucket (Figure \ref{fig:bitbucket_atmos}) and is now under the
adminstrative control of the Galactic Center Group. We have
installation notes that include dependencies (cfitsio, fftw-3.3,
plplot, doxygen, arroyo, boost, jpegsrc). Further more, M. Britton has
provided a series of test and example scripts for C++, IDL, and
command line calls to the atmospheric code. 

The Starfinder 2.0 repository was written using Starfinder v1.6 as a
baseline and the following modifications have been incorporated and
can be enabled through a simple series of flags on the IDL command
line:
\begin{itemize}
\item
\end{itemize}
In addition to the baseline Starfinder 2.0 product, we have also
added several wrapper scripts which make up our default Galactic
Center analysis pipeline. These wrapper scripts are essential to the
automated analysis of our large GC data sets and will serve as a
starting point for other users of AO Optimization and Starfinder 2.0.



\begin{itemize}
\item git repository
\item installation at IfA and Keck
\item installation documentation
\item unit tests, regression tests, and examples
\end{itemize}

\subsection{Beyond AO Optimization}

On-axis PSF reconstruction - just a few sentences.

\textbf{Extending AO Optimization to OSIRIS}
\label{sec:osiris}

Based on the results of the NIRC2 AO optimization, we are also working on extending this technique to the OSIRIS imager and spectrograph. 
