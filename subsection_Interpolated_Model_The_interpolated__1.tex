\subsection{Final Aberration Maps from Zernike Decomposition}
\label{sec:inst_zernike}

A final map of the NIRC2 field-dependent instrumental aberrations is 
constructed from the ``old'' grid of differential phase maps using Zernike decomposition and interpolation.  The total grid was first divded into two 
data sets: "on-grid" points that are spaced $\sim$100 pixels apart in x and y, and "off-grid" points at intermediate positions. These two sets of data are depicted in Figure \ref{fig:inst_on_off_grid}. The "on-grid" points were decomposed into the first 200 Zernike terms and the spatial variation of each Zernike term was modeled using a third order spline interpolation. The "off-grid" points were compared with the resulting model. Figure \ref{fig:inst_phase_diff_zernike} shows the off-grid Zernike terms are well-modeled, with the exception of defocus and astigmatism terms at the off-grid points that are the most distant from the on-grid points. 

After this analysis, the entire ``old'' data set of 150 ``on-grid'' and ``off-grid'' positions was combined to with the same methodology as above to produce a final model of the interpolated Zernike maps.
The model was used to produce a new fine grid of wavefront aberration maps with 204 $\times$ 204 grid points across the field of view, corresponding to a 5-pixel spacing in $x$ and $y$. 
The final ``old'' map was delivered to the AO Optimization team. This will be used in \textit{StarFinder} 2.0.
  
  