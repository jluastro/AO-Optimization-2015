\section{Introduction}

High-resolution observations of stars orbiting close to our Galaxy's supermassive black hole will enable unique tests of general relativity in the strong gravity regime. Adaptive optics technology currently provides the sharpest infrared view of the Galactic Center \cite{Ghez_2005}. However, the potential of this technology has not been fully exploited in terms of astrometric and photometric precision due to our incomplete knowledge of the temporal and spatial variations in the point spread function (PSF). In 2011, we embarked on a project to optimize the extraction of  

{\bf TO DO in Introduction:}
\begin{enumerate}
\item Galactic Center Science Objective
\item PSF Variability as a primary limitation (if you add spatially variable PSF, you get a drift of the reference frame)
\end{enumerate}


The following report is written for the External Advisory Committee (EAC) visit in July, 2015. This report contains a status update on the AO Optimization project including completed milestones in Year 4 (2014-07-01 to 2015-07-01) and upcoming milestones for the final Year 5. As a reminder, the AO Optimization project consists of the following high-level goals:
\begin{enumerate}
\item Map NIRC2 instrumental aberrations.
\item Develope algorithms to predict atmospheric-induced aberrations from MASS/DIMM data.
\item Update NIRC2 distortion solution.
\item Upgrade Starfinder 2.0 to extract on-axis PSF from image with spatially varying PSF.
\item Upgrade Starfinder 2.0 to extract stellar astrometry and photometry using varying PSFs.
\item Apply AO Optimization to Galactic Center astrometry.
\item Disseminate AO Optimization to a broader community.
\end{enumerate}




    
    
    
    
    
    
    
  