\section{Instrumental Modeling}
Our efforts with the instrumental modeling techniques continued forth this year. We primarily focused on characterizing time variability, acquiring additional data post-hardware adjustments, and creating an interpolated instrumental model that delivered a grid of phase maps to the entire team. 
\subsection{Time Variability}
We have taken a substantial amount of phase diversity data on NIRC2 since October 2012 (see full grid below). One of the main objectives for taking such a large amount of data over time is to measure any contribution to time variability at the four corners (hereafter called the "extrema" points) of the detector to see how the $\Delta$RMS changes as a function of time. To do this, we take data at the center of the detector and at the four corners, taking the same positions across multiple epochs. We center-subtract each of the four corners in a respective epoch, and we measure the RMS between a given epoch and the October 2012 epoch. That distribution is shown in the figure below.
    
    
    
    
    
    
    
    
    
  