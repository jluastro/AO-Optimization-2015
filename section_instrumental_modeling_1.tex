\section{Instrumental Modeling}
\label{sec:instrument}

In Year 4, we produced a final map for the NIRC2 instrumental
aberrations for data taken prior to April 2015 and delivered this map
throughout the AO Optimization team. More than two years of
phase diversity data were used to analyze the short and long term time
stability and wavelength dependence (\S\ref{sec:instrument_time}).
The Keck \textrm{II} AO system and NIRC2 were re-aligned and
re-configured in April 2015 to improve the PSF at L-band
(\S\ref{sec:instrument_hardware}). To account for these hardware changes, we 
obtained a new phase diversity data set with an even denser spacing on
the NIRC2 focal plane to make a new map of the instrumental
aberrations.
Both the ``old'' aberration map and the ``new'' aberration map were
constructed using Zernike decomposition and interpolation as has been
suggested by the EAC (\S\ref{sec:inst_zernike}).

\subsection{Time Variability and Wavelength Dependence}
\label{sec:instrument_time}

We have monitored the NIRC2 instrumental aberrations by 
acquiring phase diversity data over the entire NIRC2 field of view
multiple times between October 2012 and May 2015. 
A dense grid was first obtained in 2012 (Figure \ref{fig:inst_phase_masp}). 
Additionally, we have routinely measured the center and four courners
of the detector
(hereafter called the "extrema" points) in order to characterize the
time variability. The center phase map is subtracted from each of the
corners to produce differential phase maps. These extreme typically
show a RMS wavefront error of 190 nm. The differential phase
maps at each epoch are compared with the October 2012 and the
difference is quantified by the $\Delta$RMS as shown in Figure
\ref{fig:inst_time_var}. 
While the field-dependence of the instrumental aberrations appear to
change with time, the maximum contribution is $\sim$60 nm RMS, which
is a small fraction of the total wavefront error. 
    
    
    
  
  
  
  
  
  