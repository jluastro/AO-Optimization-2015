\section{Instrumental Modeling}
\label{sec:instrument}

In Year 4, we produced a final map for the NIRC2 instrumental
aberrations for data taken prior to April 2015 and delivered this map
throughout the AO Optimization team. More than two years of
phase diversity data were used to analyze the short and long term time
stability (\S\ref{sec:instrument_time}) and wavelength dependence 
(\S\ref{sec:instrument_chromatic}). 
The Keck \textrm{II} AO system and NIRC2 were re-aligned and
re-configured in April 2015 to improve the PSF at L-band
(\S\ref{sec:instrument_hardware}). To account for these hardware changes, we 
obtained a new phase diversity data set with an even denser spacing on
the NIRC2 focal plane to make a new map of the instrumental
aberrations.
Both the ``old'' aberration map and the ``new'' aberration map were
constructed using Zernike decomposition and interpolation as has been
suggested by the EAC.

\subsection{Time Variability}
\label{sec:instrument_time}
We have taken a substantial amount of phase diversity data on NIRC2
since October 2012 (Figure \ref{fig:inst_phase_masp}). One of the main objectives
for taking such a large amount of data over time is to measure any
contribution to time variability at the four corners (hereafter called
the "extrema" points) of the detector to see how the $\Delta$RMS
changes as a function of time. To do this, we take data at the center
of the detector and at the four corners, taking the same positions
across multiple epochs. We center-subtract each of the four corners in
a respective epoch, and we measure the RMS between a given epoch and
the October 2012 epoch as shown in Figure \ref{fig:inst_time_var}. 
Since this time variability analysis is complete for the full grid,
there is at maximum a contribution of $\sim$60 nm RMS from time
variability at the extreme points. Note that this is still far lower
than the measured 190 nm RMS wavefront error measured at the extrema.


    
    
    
    
    
  
  
  
  