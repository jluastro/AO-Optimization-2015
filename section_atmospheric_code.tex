\section{Atmospheric Code}
\label{ref:atmospheric_code}

The atmospheric code delivered by Matthew Britton in Year 3 (2013-2014) continues to be tested in house at UCLA. Installation at Keck and at the IfA has been attempted and yielded new installation documentation that will be essential during the dissemination stage of the project (EAC-6). However, theGiven M. Britton’s departure from tOSC, the atmospheric code has been ported into a bitbucket hosted git code repository for easy access and management by multiple AO Optimization team members. As a result of this migration, we have lost the automated testing capabilities that the Jenkins framework (used at tOSC) afforded; however, we view this a small risk since we can run the tests manually. While extensively exercising the atmospheric code, we have also tested and verified the expected geometry and coordinate systems for the position of the desired PSF, the laser guide star (LGS), and the tip-tilt guide star (TTGS) as shown in Figure 1. We have also confirmed that performance degrades with separation from the LGS and the TTGS, again as expected and shown in Figure 2 (EAC-2a, EAC-2b).  