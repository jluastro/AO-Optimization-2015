The Starfinder 2.0 repository was written using Starfinder v1.6 as a
baseline and the following modifications have been incorporated and
can be enabled through a simple series of flags on the IDL command
line:
\begin{itemize}
\item
\end{itemize}
In addition to the baseline Starfinder 2.0 product, we have also
added several wrapper scripts which make up our default Galactic
Center analysis pipeline. These wrapper scripts are essential to the
automated analysis of our large GC data sets and will serve as a
starting point for other users of AO Optimization and Starfinder 2.0.

\begin{itemize}
\item talks? 
\end{itemize}

\subsection{Beyond AO Optimization}

The AO Optimization project is connected to two additional PSF
reconstruction projects. The Keck on-axis PSF-reconstruction (PSF-R)
project, funded through and NSF ATI grant (PI: P. Wizinowich), has
been on a parallel track
to AO Optimization.  PSF-R aims to reconstruct the on-axis PSF from AO
wave-front sensor telemetry for both NGS and LGS observations. 
The success of the AO Optimization project, so far, and the desire to
combine both the on-axis and off-axis PSF estimation techniques led us
to submit a follow-up NSF ATI proposal to build a Keck PSF
reconstruction facility that delivers grids of PSFs for all Keck AO
instruments. This proposal (co-PIs: Wizinowich, Ghez, Lu) was submitted 
in Fall 2014 and is still pending. 

The AO Optimization project for NIRC2 has also inspired an extension
to the OSIRIS imager and spectrograph, which has been successfully
funded by the Moore Foundation. YYY MORE? YYY

