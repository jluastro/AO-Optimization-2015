The Starfinder 2.0 repository was written using Starfinder v1.6 as a
baseline and the following modifications have been incorporated and
can be enabled through a simple series of flags on the IDL command
line:
\begin{itemize}
\item
\end{itemize}
In addition to the baseline Starfinder 2.0 product, we have also
added several wrapper scripts which make up our default Galactic
Center analysis pipeline. These wrapper scripts are essential to the
automated analysis of our large GC data sets and will serve as a
starting point for other users of AO Optimization and Starfinder 2.0.

\begin{itemize}
\item talks? 
\end{itemize}

\subsection{Beyond AO Optimization}

The AO Optimization project is connected to two additional PSF
reconstruction projects: the Keck on-axis PSF-reconstruction project
and the Moore OSIRIS Upgrade and AO Optimization. 

The Keck on-axis PSF-reconstruction (PSF-R) project is 
funded through an NSF ATI grant (PI: P. Wizinowich) and has
been on a parallel schedule to AO Optimization.  
PSF-R aims to reconstruct the on-axis PSF from AO
wave-front sensor telemetry for both NGS and LGS observations. 
The success of the AO Optimization project, so far, and the desire to
combine both the on-axis and off-axis PSF estimation techniques led us
to submit a follow-up NSF ATI proposal to build a Keck PSF
reconstruction facility that delivers grids of PSFs for all Keck AO
instruments. This proposal (co-PIs: Wizinowich, Ghez, Lu) was submitted 
in Fall 2014 and is still pending. 

The Moore OSIRIS upgrade and AO Optimization project
was inspired by the WMKF AO Optimization project and has been funded
by the Moore Foundation. 
The goal of this project is to increase the scientific output of
OSIRIS by upgrading the imaging camera and extending the AO
Optimization methodology to estimate the spectrograph PSF from imager
observations of stars. This project is in its second year.
The imager upgrade has well underway; preliminary design reviews
have been completed \textbf{(YYY ??? YYY)} and detailed design work is
commencing.
On the methodology front, we have confirmed that AO Optimization can
succesfully predict PSFs at a separation of over 20'', which is the
separation between the center of the imager and spectrograph.
Finally, preliminary software has been implemented to extract OSIRIS
spectra using Starfinder with a non-wavelength dependendent, single
PSF. The next step depends on the completion of Starfinder 2.0 from
the WMKF AO Optimization project. 




