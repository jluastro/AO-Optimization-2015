\section{Project Goals and Status}
\label{sec:overview}

Over the past year, the AO Optimization project has revised and
updated the original project goals based on our progress and
experience in the first three years of the project 
(Figure \ref{fig:ao_opt_schematic}). We view the
following seven top-level objectives as essential for completion of
the AO Optimization project:
\begin{enumerate}
\item[Goal 1.] Develop algorithms to predict atmospheric-induced aberrations
  from MASS/DIMM data for both natural and laser guide star observations.
\item[Goal 2.] Map NIRC2 instrumental aberrations that effect the PSF shape.
\item[Goal 3.] Update the NIRC2 distortion solution utilizing a spatially
  variable PSF.
\item[Goal 4.] Upgrade Starfinder 2.0 to extract an on-axis PSF from images with
  a spatially varying PSF due to atmospheric and instrumental aberrations.
\item[Goal 5.] Upgrade Starfinder 2.0 to extract stellar astrometry and
  photometry using spatially varying PSFs.
\item[Goal 6.] Apply AO Optimization tools to Galactic Center and demonstrate
  astrometric gains.
\item[Goal 7.] Disseminate AO Optimization tools to a broader community.
\end{enumerate}
These revised goals differ from the original project goals 
(Appendix \ref{app:original_goals}) in that
they are more specific and better represent the software and
algorithm development (atmospheric code and Starfinder 2.0) conducted after Year
3. 

% As a reminder, the AO Optimization project funded by the
% W.M.K.F. supports YYY HOW MANY POSTDOCS YYYY with additional unfunded 
% contributions by YYY OTHER PERSONNEL YYY. 

For each of these goals, we briefly present the status at the end of
Year 3 and a summary of what has been accomplished in Year 4. More
details on the Year 4 results are then presented in subsequent
sections.


\subsection{Goal 1: Develop algorithms to predict atmospheric-induced aberrations
  from MASS/DIMM data for both natural and laser guide star observations.}

\noindent
\textit{Year 3 Status:}
The atmospheric code was delivered by Matthew Britton and
implemented both NGS and LGS off-axis PSF estimation. This code was
accessed at UCLA from a git repository and a Jenkins continuous
integration server, including unit tests, hosted at tOSC
(M. Britton's employer). A partially functional version of the code
was also installed at Keck and a web interface was 
provided\footnote{\url{https://jenkins.tosc.com}}.
The atmospheric code was tested on a suite of NGS and LGS
binary stars as well as indpendent theoretical predictions and, in
all cases, the predicted off-axis PSFs from the atmospheric code
matched the comparison off-axis PSF very well with respect to
Strehl, FWHM, and PSF residuals. 

\noindent
\textit{Year 4 Status:} (see also \S\ref{sec:atmospheric_code})
During year 4, the atmospheric code has been routinely used with on-sky data with good success and is nearly complete.
We identified a number of small bugs and one major
performance issue with the IDL interface to the atmospheric code, 
which is used by Starfinder 2.0. The interface has been re-designed, but not yet
implemented, to improve Starfinder 2.0 efficiency by 10-100
fold. Additionally, the atmospheric code repository has migrated away
from tOSC due to M. Britton's change of employment. 
UCLA now administers the git repository.



\subsection{Goal 2:  Map NIRC2 instrumental aberrations that effect the PSF
    shape.}

\noindent 
\textit{Year 3 Status:}
The instrumental aberrations that effect PSF shape were measured over
a dense grid in the NIRC2 focal plane using in-focus and out-of-focus
fiber observations and phase diversity techniques. The resulting
discrete grid of phase maps was made continuous using nearest-neighbor
interpolation. The resulting aberration maps exhibited temporal
variations of $\sim$50 nm over 1.5 years, which is significantly
larger than the measurement error of $\sim$15 nm. We also reported 
chromatic differences between the Br-$\gamma$, Fe \textrm{II}, and Kp
filters of 50$-$80 nm. The variations appeared to be due to a
wavelength-dependent defocus term.
  
\noindent 
\textit{Year 4 Status:} (see also \S\ref{sec:instrument})
We have moved to using Zernike decomposition of the phase maps and 
interpolation of the individual
Zernike coefficients to produce a continuous aberration map. 
We have extended the time baseline of our instrumental aberration
monitoring campaign to over 2.5 years (Oct 2012 - June 2015). This
period also includes a major hardware re-alignment of the NIRC2
system, which required a complete new set of phase diversity
measurements. The "old" aberration maps are complete and have been 
delivered to the rest of the AO Optimization team. Data sets for 
the "new" aberration maps are 70\% complete.

\subsection{Goal 3: Update the NIRC2 distortion solution utilizing a spatially
  variable PSF.}

\noindent
\textit{Year 3 Status:}
No new progress reported in Year 3.

\noindent
\textit{Year 4 Status:} (see also \S\ref{sec:distortion})
We have successfully observed the globular cluster M53 using NIRC2
while simultaneously measuring the turbulence profile with the
MASS/DIMM unit. We have completed an extensive dither/rotation pattern
that will allow us to extract a new NIRC2 distortion solution using
Starfinder 2.0.


\subsection{Goal 4: Upgrade Starfinder 2.0 to extract an on-axis PSF from images with
  a spatially varying PSF due to atmospheric and instrumental aberrations.}

\noindent
\textit{Year 3 Status:} 
We presented preliminary results from our integration of the
atmospheric code with Starfinder, which became the foundation for
Starfinder 2.0. We validated that this prototype could extract an
on-aixs (un-observable) PSF from a set of off-axis guide stars and a
MASS/DIMM measurement. 

It is correct that the tools I presented were integrated with
Starfinder. But what I did not realize at that point is how much of
our way to do PSF-extraction and -fitting is actually implemented in
our find\_stf wrappers. The real strategy of our analysis is there,
not in Starfinder. So we spent now most of the time to rigorously
integrate it with our software that is the result of a decade of
development. That is way more useful than having a handful of
primitive new wrappers, and it is a whole different level of
complication, especially if the goal is to preserve the previous
functionality. We also improved Starfinder beyond the AO-opt part by
including many changes of the versions that were floating around. 

\noindent
\textit{Year 4 Status:}
(see also \S\ref{sec:astrometric_theory} and \S\ref{sec:tools})
The atmospheric code is now fully integrated into Starfinder 2.0 and
PSFs can be extracted in an iterative fashion using our standard
(upgraded) analysis wrapper code. 


\subsection{Goal 5: Upgrade Starfinder 2.0 to extract stellar astrometry and
  photometry using spatially varying PSFs.}

\noindent
\textit{Year 3 Status:}
No new progress was reported in Year 3.

\noindent
\textit{Year 4 Status:}
(see also \S\ref{sec:astrometric_theory} and \S\ref{sec:tools})
The atmospheric code is now fully integrated into Starfinder 2.0 and
stellar astrometry and photometry can be extracted in an iterative
fashion using our standard
(upgraded) analysis wrapper code. 


\subsection{Goal 6: Apply AO Optimization tools to Galactic Center and demonstrate
  astrometric gains.}

\noindent
\textit{Year 3 Status:}
We presented a test run using our first-pass version of Starfinder
2.0 on a single maser mosaic image. While this test was simple and highly
customized as compared with our normal Galactic Center
analysis, it was an essential first step in demonstrating that the
``on-axis'' PSF could be extracted from on-sky Galactic Center data.
 
\noindent
\textit{Year 4 Status:}
(see also \S\ref{sec:testing})
With the completion of Starfinder 2.0, which calls the atmospheric
code, we have begun testing and optimizing with on-sky data. We have
successfully run Starfinder 2.0 on a single-night stack of Galactic
Center images. PSF extraction and prediction has been validated
with analysis of image residuals, astrometry, and photometry compared
with the original Starfinder 1.6. 


\subsection{Goal 7: Disseminate AO Optimization tools to a broader community.}

\noindent
\textit{Year 3 Status:}
The atmospheric code was deployed to the public via a web-service
interface hosted at Keck. This web-service allowed users to model
the atmospheric effects on PSFs throughout a field of view. 

\noindent
\textit{Year 4 Status:}
(see also \S\ref{sec:dissemination})
All of the AO Optimization software has been organized and archived in
a project code repository available on bitbucket.org. We have
documented the installation process and new modules in Starfinder
2.0. We have developed a suite of tests and code examples for both the
atmospheric code, Starfinder 2.0, and the . This extensive
organization effort now enables the AO Optimization code-base to be
distributed easily and worked on by several team members at once,
greatly increasing the efficiency of our development effort and laying
a solid foundation for future dissemination to Keck AO users.


  
  
  
  