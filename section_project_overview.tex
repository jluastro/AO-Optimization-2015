\section{Project Overview and Status}
\label{sec:overview}

Over the past year, the AO Optimization project has revised and
updated the original project goals based on our progress and
experience in the first three years of the project. We view the
following seven top-level objectives as essential for completion of
the AO Optimization project:
\begin{enumerate}
\item Develop algorithms to predict atmospheric-induced aberrations
  from MASS/DIMM data for both natural and laser guide star observations.
\item Map NIRC2 instrumental aberrations that effect the PSF shape.
\item Update the NIRC2 distortion solution utilizing a spatially
  variable PSF.
\item Upgrade Starfinder 2.0 to extract an on-axis PSF from images with
  a spatially varying PSF due to atmospheric and instrumental aberrations.
\item Upgrade Starfinder 2.0 to extract stellar astrometry and
  photometry using spatially varying PSFs.
\item Apply AO Optimization tools to Galactic Center and demonstrate
  astrometric gains.
\item Disseminate AO Optimization tools to a broader community.
\end{enumerate}
These revised goals differ from the original project goals 
(\ref{app:original_goals}) in that
they reflect the time and effort needed for software and
algorithm development (atmospheric code and Starfinder 2.0) after Year
3. As a reminder, the AO Optimization project funded by the
W.M.K.F. supports YYY HOW MANY POSTDOCS YYYY with additional unfunded 
contributions by YYY OTHER PERSONNEL YYY. 

\subsection{End of Year 3 Status}
At the end of Year 3, we reported to the EAC the following status for
each of the top-level goals:
\begin{enumerate}
\item {\em Develop algorithms to predict atmospheric-induced aberrations
  from MASS/DIMM data for both natural and laser guide star
  observations.} \\
  The ``atmospheric code``was delivered by Matthew Britton that
  implemented both NGS and LGS off-axis PSF estimation. This code was
  accessed at UCLA by way of a git repository and a Jenkins continuous
  integration server, including unit tests, hosted at tOSC
  (M. Britton's employer). A version of the code was also installed at
  Keck and a web interface has been provided at YYY INSERT URL YYY.

  The atmospheric code was tested on a suite of NGS and LGS
  binary stars as well as indpendent theoretical predictions and, in
  all cases, the predicted off-axis PSFs from the atmospheric code
  matched the comparison off-axis PSF very well with respect to
  Strehl, FWHM, and PSF residuals. 

\item {\em Map NIRC2 instrumental aberrations that effect the PSF
    shape.} \\
  The instrumental aberrations that effect PSF shape were
  measured using a grid of in/out-of focus fiber observations and
  phase diversity techniques. The resulting phase map YYY \\
  temporal stability \\
  chromaticity 
  
\item {\em Update the NIRC2 distortion solution utilizing a spatially
  variable PSF.} \\
  YYY No data as of last year. YYY

\item {\em Upgrade Starfinder 2.0 to extract an on-axis PSF from images with
  a spatially varying PSF due to atmospheric and instrumental
  aberrations.} \\
  YYY Had a test case using STF 2.0 under-pinnings to extract an
  on-axis PSF. However, this wasn't implemented within our STF 2.0
  framework and wrappers. Here is Gunther's description of the
  differences. 

  It is correct that the tools I presented were integrated with
  Starfinder. But what I did not realize at that point is how much of
  our way to do PSF-extraction and -fitting is actually implemented in
  our find\_stf wrappers. The real strategy of our analysis is there,
  not in Starfinder. So we spent now most of the time to rigorously
  integrate it with our software that is the result of a decade of
  development. That is way more useful than having a handful of
  primitive new wrappers, and it is a whole different level of
  complication, especially if the goal is to preserve the previous
  functionality. We also improved Starfinder beyond the AO-opt part by
  including many changes of the versions that were floating around. 


\item {\em Upgrade Starfinder 2.0 to extract stellar astrometry and
  photometry using spatially varying PSFs.} \\
  YYY No progress in year 3?

\item {\em Apply AO Optimization tools to Galactic Center and demonstrate
  astrometric gains.} \\
  YYY Gunther presented a test on a single maser mosaic image. It used
  the under-pinnings of STF 2.0; but not our official ``wrapper'' of
  STF 2.0.

\item {\em Disseminate AO Optimization tools to a broader community.} \\
  The atmospheric code was deployed to the public via a web-service
  interface hosted at Keck. This web-service allowed users to model
  the atmospheric effects on PSFs throughout a field of view. 
\end{enumerate}

YYY Any comments about personnel changes here? We commented in the
mid-term report.

\subsection{End of Year 4 Status}
\begin{enumerate}
\item {\em Develop algorithms to predict atmospheric-induced aberrations
  from MASS/DIMM data for both natural and laser guide star
  observations.} \\
  YYY Code has migrated away from tOSC due to the departure of
  M. Britton. UCLA has taken ownership of the git repository (but not
  the Jenkins server). 

  During year 4, we identified a number of small bugs and one major
  performance issue with the interface between Starfinder 2.0 and the
  atmospheric code. The interface has been re-designed, but not yet
  implemented, to improve Starfinder 2.0 efficiency by 10-100
  fold. More details are presented in \S\ref{sec:atmospheric_code}.

\item {\em Map NIRC2 instrumental aberrations that effect the PSF
    shape.} \\
  
  
\item {\em Update the NIRC2 distortion solution utilizing a spatially
  variable PSF.} \\


\item {\em Upgrade Starfinder 2.0 to extract an on-axis PSF from images with
  a spatially varying PSF due to atmospheric and instrumental
  aberrations.} \\


\item {\em Upgrade Starfinder 2.0 to extract stellar astrometry and
  photometry using spatially varying PSFs.} \\


\item {\em Apply AO Optimization tools to Galactic Center and demonstrate
  astrometric gains.} \\

\item {\em Disseminate AO Optimization tools to a broader community.} \\
\end{enumerate}

  
  