\appendix{Original WMKF Goals and Timeline}
The original project goals in the WMKF proposal are listed below. Goals that are completed, in progress, and incomplete are presented in green, yellow, and green, respectively.

\begin{itemize}
\item[1] Dramatically improve measurements obtained through Adaptive Optics (AO) hardware by developing the first methodology to determine spatial variations in image quality for AO systems on large (8-10 m) ground-based telescopes, using the premier such system at the W. M. Keck Observatory. This methodology, which we call “AO-Optimization”, will be carried out for both Natural Guide Star AO and Laser Guide Star AO. With factors of 4 to 100 improvements expected from measurements at the center and edges of the AO images, respectively, AO- Optimization greatly enhances the investments made to date in AO hardware at Keck and elsewhere.
    \begin{itemize}
    \item[1-1] Develop \& test AO data analysis tools to work with variable point spread function (PSF)
    \item[1-2] Static (instrumental) PSF variations
        \begin{itemize}
        \item[1-2a] Carry out lab tests with AO fiber to measure static PSF
        \item[1-2b] Derive model for static PSF variations from fiber data 
        \item[1-2c] Test, optimize and characterize static PSF model
        \end{itemize}
    \item[1-3] Natural Guide Star (NGS) AO PSF variations
        \begin{itemize}
        \item[1-3a] Develop code to predict PSF variations for NGS-AO observations
            using Mauna Kea atmospheric profiler data, extending the proof of concept 
            work done at a smaller telescope to a larger telescope on a different site.
        \item[1-3b] Collect testing data from Keck NGS-AO and download associated MASS/DIMM data
        \item[1-3c] Test, optimize and characterize NGS-AO PSF model
        \end{itemize}
    \item[1-4] Laser Guide Star (LGS) AO PSF variations
        \begin{itemize}
        \item[1-4a] Develop code to predict PSF variations for LGS-AO observations
            using Mauna Kea atmospheric profiler data
        \item[1-4b] Collect testing data from Keck LGS-AO system and download
            associated MASS/DIMM data
        \item[1-4c] Test, optimize and characterize LGS-AO PSF model
        \end{itemize}
    \item[1-5] Evaluation: Assessment by external project advisory committee at year end 1-5a. 
    Assemble Committee at beginning of Year 1
        \begin{itemize}
        \item[1-5b] Review static instrumental PSF modeling 
        \item[1-5c] Review NGS-AO PSF modeling
        \item[1-5d] Review LGS-AO PSF modeling
        \end{itemize}
    

    \end{itemize}
\item
\item
\end{itemize}
    
  
  