\appendix
\section{Original WMKF Goals and Timeline}
\label{app:original_goals}
The original project goals in the WMKF proposal are listed below. Goals that are completed, in progress, and incomplete are presented in green, yellow, and green, respectively.

\begin{itemize}
\item[1.] Dramatically improve measurements obtained through Adaptive Optics (AO) hardware by developing the first methodology to determine spatial variations in image quality for AO systems on large (8-10 m) ground-based telescopes, using the premier such system at the W. M. Keck Observatory. This methodology, which we call “AO-Optimization”, will be carried out for both Natural Guide Star AO and Laser Guide Star AO. With factors of 4 to 100 improvements expected from measurements at the center and edges of the AO images, respectively, AO- Optimization greatly enhances the investments made to date in AO hardware at Keck and elsewhere.
    \begin{itemize}
    \item[1-1.] Develop \& test AO data analysis tools to work with variable point spread function (PSF)
    \item[1-2.] Static (instrumental) PSF variations
        \begin{itemize}
        \item[1-2a.] Carry out lab tests with AO fiber to measure static PSF
        \item[1-2b.] Derive model for static PSF variations from fiber data 
        \item[1-2c.] Test, optimize and characterize static PSF model
        \end{itemize}
    \item[1-3.] Natural Guide Star (NGS) AO PSF variations
        \begin{itemize}
        \item[1-3a.] Develop code to predict PSF variations for NGS-AO observations
            using Mauna Kea atmospheric profiler data, extending the proof of concept 
            work done at a smaller telescope to a larger telescope on a different site.
        \item[1-3b.] Collect testing data from Keck NGS-AO and download associated MASS/DIMM data
        \item[1-3c.] Test, optimize and characterize NGS-AO PSF model
        \end{itemize}
    \item[1-4.] Laser Guide Star (LGS) AO PSF variations
        \begin{itemize}
        \item[1-4a.] Develop code to predict PSF variations for LGS-AO observations
            using Mauna Kea atmospheric profiler data
        \item[1-4b.] Collect testing data from Keck LGS-AO system and download
            associated MASS/DIMM data
        \item[1-4c.] Test, optimize and characterize LGS-AO PSF model
        \end{itemize}
    \item[1-5.] Evaluation: Assessment by external project advisory committee at year end
        \begin{itemize}
        \item[1-5a.] Assemble Committee at beginning of Year 1
        \item[1-5b.] Review static instrumental PSF modeling 
        \item[1-5c.] Review NGS-AO PSF modeling
        \item[1-5d.] Review LGS-AO PSF modeling
        \end{itemize}
    \end{itemize}
\item[2.] Apply AO-Optimization to W. M. Keck Observatory AO observations
  of the Galactic Center to improve measurements of stellar orbits
  around the supermassive black hole at the center of our Galaxy. This
  will enable us to test Einstein’s theory of General Relativity (GR)
  and black hole growth models (extended dark mass distribution) in
  regimes that have never been probed before.
  \begin{itemize}
  \item[2-1.] Collect data on globular clusters from Keck NGS-AO system
    for modeling the geometric optical distortion of W. M. Keck
    Observatory’s AO imaging camera (NIRC2) and download associated
    MASS/DIMM data.
  \item[2-2.] Derive a new model for the geometric optical distortion
    of NIRC2, incorporating static and NGS-AO PSF variation models.
  \item[2-3.] Collect data on Galactic center from Keck LGS-AO system
    and download associated MASS/DIMM data
  \item[2-4.] Derive new Galactic center reference frame with new
    distortion model
  \item[2-5.] Improve Galactic center positional precision of SgrA*,
    S0-2, and other short-period stars by incorporating static and
    LGS-AO PSF variation model
  \item[2-6.] Test impact of work on sensitivity of S0-2’s orbital
    measurement to GR and black hole growth models with current and
    future data sets
    \begin{itemize}
    \item[2-6a.] Improved reference frame from new distortion model
    \item[2-6b.] Improved reference frame from adding improved SgrA*
      position measurement
    \item[2-6c.] Improved measurements of other short-period stars
    \end{itemize}
  \item[2-7.] Evaluation: Assessment by external project advisory
    committee 
    \begin{itemize}
    \item[2-7a.] Review NIRC2 distortion modeling
    \item[2-7b.] Review reference frame construction
    \item[2-7c.] Review orbital motion modeling
    \end{itemize}
  \end{itemize}
\item[3.] Enable the application of AO-Optimization across the field of
  Astronomy. This will maximize the scientific return of Adaptive
  Optics at W. M. Keck Observatory as well as at other current and
  future AO facilities around the world, and will enable new
  scientific discoveries well beyond the Galactic Center.
  \begin{itemize}
  \item[3-1.] Package, publish, and disseminate PSF model
    \begin{itemize}
    \item[3-1a.] Static (instrumental) PSF component
    \item[3-1b.] NGS-AO PSF model
    \item[3-1c.] LGS-AO PSF model
    \end{itemize}
  \item[3-2.] Present demonstrated benefit of methodology at key
    meetings for AO specialists (e.g., SPIE, Optical Society of
    America) and Astronomers at large (e.g., American Astronomical
    Association)
    \begin{itemize}
    \item[3-2a.] Static (instrumental) PSF component
    \item[3-2b.] NGS-AO PSF model
    \item[3-2c.] LGS-AO PSF model
    \end{itemize}
  \item[3-3.] Seek meetings with leaders of future generation AO
    initiatives to discuss
    \begin{itemize}
    \item[3-3a.] Synergies
    \item[3-3b.] Results and implications of work NGS-AO PSF model 
    \item[3-3c.] Results and implications of work LGS-AO PSF model
    \end{itemize}
  \item[3-4.] Package, publish, and disseminate AO data analysis tools
    to work with variable PSF
  \item[3-5.] Package, publish, and disseminate new distortion solution
  \item[3-6.] Assess and implement noteworthy public relations angles
  \item[3-7.] Evaluation: Assessment by external project advisory
    committee at year end 
    \begin{itemize}
    \item[3-7a.] Review static instrumental PSF dissemination
    \item[3-7b.] Review NGS-AO PSF dissemination
    \item[3-7c.] Review AO data analysis too dissemination
    \item[3-7d.] Review distortion solution dissemination
    \item[3-7e.] Review LGS-AO PSF dissemination
    \end{itemize}
  \end{itemize}
\end{itemize}
    

\section{EAC Recommendations from 2014 August}  
\label{app:eac_recommend}

The following recommendations were extracted from the 2014 August EAC
report and are referred to throughout the text of this document.

\begin{itemize}
\item[EAC 1.] NIRC2 Aberration Maps
  \begin{itemize}
  \item[EAC 1-1.] Explore Focus vs. wavelength
  \item[EAC 1-2.] Decompose phase maps into Zernikes, mitigate high spatial
    frequency residuals
  \end{itemize}
\item[EAC 2.] Simulations to Address Questions on Atmospheric Code
  \begin{itemize}
  \item[EAC 2-1.] Does the code accurately predict the PSF variation
    vs. distance to tip-tilt star?
  \item[EAC 2-2.] Does the code accurately predict the PSF variation
    vs. distance to the LGS?
  \item[EAC 2-3.] Does the outer scale (not included in atmosphere code) matter
    for the GC?
  \end{itemize}
\item[EAC 3.] Starfinder 2.0
  \begin{itemize}
  \item[EAC 3-1.] examine astrometric error vs. Cn2 profile develop algorithm to
    incorporate prior information on sources, what are the advantages?
  \end{itemize}
\item[EAC 4.] Simulations
  \begin{itemize}
  \item[EAC 4-1.] simulate crowded fields to test STF 2.0
  \item[EAC 4-2.] use simulations to update astrometric error budget and show
    improvements from AO Optimization
  \end{itemize}
\item[EAC 5.] Test STF 2.0 on Data
  \begin{itemize}
  \item[EAC 5-1.] archival NIRC2 data sets that are less crowded to verify algorithm
  \item[EAC 5-2.] use GC multi-epoch analysis to determine improvements from
    reducing false detections.
  \end{itemize}
\item[EAC 6.] Improve documentation and organization
  \begin{itemize}
  \item[EAC 6-1.] Review original goals and revise/prioritize project milestones
    mitigate personnel change-overs
  \end{itemize}
\item[EAC 7.] Dissemination
  \begin{itemize}
  \item[EAC 7-1.] identify and share with alpha-test team
  \end{itemize}
\end{itemize}
