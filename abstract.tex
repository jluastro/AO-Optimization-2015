\textbf{ABSTRACT:} The AO Optimization project, funded by the W. M. Keck Foundation and UCLA, has completed its 4th year. The following report to the external advisory committee (EAC) details our progress on activities during the period from Summer 2014 (last EAC visit and report) and July 2015. Several major milestones have been reached in this period, including
\begin{itemize}
\item {\em Starfinder 2.0 is fully functional.} This tool extracts and utilizes a spatially variable point spread function (PSF) derived from modeling the atmospheric and instrumental aberrations. The spatially varying PSF is used to extract astrometry and photometry. The process runs iteratively and delivers a starlist in a nearly identical manner to previous versions of Starfinder.

\item {\em The AO Optimization tools (including Starfinder 2.0) have been tested on both simulated and observed data sets.} Initial tests on a small-dither stack of Galactic Center exposures within a single night reveal differences in star positions between a single-PSF and variable-PSF analysis of $\sim$0.4$\pm$0.4 mas. Further tests on large-dither mosaics are needed to confirm whether the new positions are more accurate.

\item {\em AO Optimization software has been cohesively organized with documentation, tests, and examples.} This extensive organization effort now enables the AO Optimization code-base to be easily distributed to all team members, greatly increasing our development efficiency, and lays a solid foundation for future dissemination to Keck AO users.

\item {\em All on-sky data have been collected to re-derive the NIRC2 distortion solution.}
The data have been fully reduced using an upgraded NIRC2 reduction pipeline that is available to the public on github.

\end{itemize}
Over the past year, the project has also undergone several changes on the personnel front. We have appointed a new project manager (Jessica Lu, IfA), a new reference frame and simulation lead (Gunther Witzel, UCLA), and new graduate students that are being trained in the areas of modeling S0-2’s orbit (Devin Chu, UCLA) and deriving distortion solutions (Max Service, IfA). Overall, the project is on track for completion at the end of Year 5.


  