\textbf{ABSTRACT:} The AO Optimization project, funded by the W. M. Keck Foundation and UCLA, has completed its 4th year. The following report to the external advisory committee (EAC) is intended as a progress report on activities during the period from Summer 2014 (last EAC visit and report) and July 2015. Several major milestones have been reached in this period, including
\begin{itemize}
\item {\em Starfinder 2.0 is fully functional.} This tool extracts and utilizes a spatially variable point spread function (PSF) derived from modeling the atmospheric and instrumental
aberrations. Further optimization and testing is still underway.
\item {\em The AO Optimization tools (including Starfinder 2.0) have been tested on both simulated and observed data sets.} Initial tests on a stack of Galactic Center exposures within a single night reveal YYYY ADD RESULTS YYYY.
\item {\em AO Optimization software has been cohesively organized with documentation, tests, and examples.} This extensive organization effort now enables the AO Optimization code-base to be easily distributed to all team members, greatly increasing our development efficiency, and lays a solid foundation for future dissemination to Keck AO users.
\item {\em All on-sky data have been collected to re-derive the NIRC2 distortion solution.}
\end{itemize}
Over the past year, the project has also undergone several changes on the personnel front. We have appointed a new project manager (Jessica Lu, IfA), a new reference frame and simulation lead (Gunther Witzel, UCLA), and new graduate students that are being trained in the areas of modeling S0-2’s orbit (Devin Chu, UCLA) and deriving distortion solutions (Max Service, IfA). 

  